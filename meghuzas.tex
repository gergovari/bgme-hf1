\section{Csavar előfeszítése és meghúzási nyomatéka}

\subsection[Csavar szabvány]{Csavar szabvány\protect\footnote{ISO 4014 szabvány alapján kapott értékek.}}
\begin{align*}
	& p = \siunit{\csavarp}{\mm} \\
	& {d_3}_\text{cs} = \siunit{\csavardthree}{\mm} \\
	& {d_2}_\text{cs} = \siunit{\csavardtwo}{\mm} \\
	& d_w = \siunit{\csavardw}{\mm} \\
	& b = \siunit{\csavarb}{\mm} \\
	& l = \siunit{\csavarl}{\mm} \\
	& \beta = \siunit{\csavarbeta}{\degree}
\end{align*}

\begin{equation*}
	\mu_{\substack{\text{min}\\\text{max}}}
	= \substack{
		\siunit{\csavarumin}{-} \\
		\siunit{\csavarumax}{-} \\
	}
\end{equation*}

\begin{center}
	\begin{tabular}{l}
		$p$: menet emelkedése \siunit{}{\mm} \\
		${d_3}_\text{cs}$: orsó magátmérője \siunit{}{\mm} \\
		${d_2}_\text{cs}$: csavar középátmérője \siunit{}{\mm} \\
		$\beta$: menetprofil szöge \siunit{}{\degree} \\
		$\mu_{\substack{\text{min}\\\text{max}}}$: súrlódási tényező\footnote{MSZ EN 24014 szabvány alapján kapott értékek.} \siunit{}{-} \\
	\end{tabular}
\end{center}

\newpage
\subsection[Meghúzási nyomaték]{Meghúzási nyomaték\protect\footnote{A feladathoz mellékelt segédletből származó számítások. (9-10. oldal)}}

$\alpha$ menetemelkedési szög számítható eddigi adatainkból. A látszólagos súrlódási félkúpszög ($\rho^{'}$) pedig az ismert súrlódási tényezőkből. A csavar meghúzásához szükséges nyomaték ($M_\text{meghúzási}$) a csavar mentén ($M_\text{csavar}$) -és az anya homlokfelületén ($M_\text{anya}$) ébredő súrlódás összege.

\begin{align}
	&\alpha = \arctan{\frac{p}{{d_2}_\text{cs} \pi}} = \siunit{\csavaralpha}{\degree} \\
	&\mu^{'}_{\substack{\text{min}\\\text{max}}}
	= \frac
		{\mu_{\substack{\text{min}\\\text{max}}}}
		{\cos{\frac{\beta}{2}}} \\
	&\rho^{'}_{\substack{\text{min}\\\text{max}}} 
	= \arctan{\mu^{'}_{\substack{\text{min}\\\text{max}}}}
	= \substack{
		\siunit{\csavarptmin}{\degree} \\
		\siunit{\csavarptmax}{\degree}
	} \\
	&d_a = \frac{d_w + M}{2} = \siunit{\csavarda}{\mm}
\end{align}

\begin{align}
	&{M_\text{csavar}}_{\substack{\text{min}\\\text{max}}} 
	= F_v \frac{{d_2}_\text{cs}}{2} \tan{\left(\alpha + {\rho^{'}}_{\substack{\text{min}\\\text{max}}}\right)} 
	= \substack{
		\siunit{\csavarMcsmin}{\newton\mm} \\
		\siunit{\csavarMcsmax}{\newton\mm}
	}\\
	&{M_\text{anya}}_{\substack{\text{min}\\\text{max}}} 
	= F_v \frac{d_a}{2} {\mu^{'}_{\substack{\text{min}\\\text{max}}}}  
	= \substack{
		\siunit{\csavarMamin}{\newton\mm} \\
		\siunit{\csavarMamax}{\newton\mm}
	}\\
\end{align}

\begin{equation}
	{M_\text{meghúzási}}_{\substack{\text{min}\\\text{max}}} = {M_\text{csavar}}_{\substack{\text{min}\\\text{max}}} + {M_\text{anya}}_{\substack{\text{min}\\\text{max}}}
	= \substack{
		\siunit{\csavarmegmin}{\newton\mm} \\
		\siunit{\csavarmegmax}{\newton\mm}
	}\\
\end{equation}

\begin{center}
	\begin{tabular}{l}
		$\alpha$: menetemelkedés szöge \siunit{}{\degree} \\
		$\mu_{\substack{\text{min}\\\text{max}}}$: súrlódási tényező \siunit{}{-} \\
		$\beta$: menetprofil szöge \siunit{}{\degree} \\
		$d_a$: anya felvekvő felület középátmérője \siunit{}{\mm} \\
		$M$: csavar szabványos mérete \siunit{}{\mm} \\
		${d_2}_\text{cs}$: menet középátmérője \siunit{}{\mm} \\
		${M_\text{csavar}}_{\substack{\text{min}\\\text{max}}}$: menet súrlódása \siunit{}{\newton\mm} \\
		$F_v$: csavar terhelése \siunit{}{\newton} \\
		$\rho^{'}_{\substack{\text{min}\\\text{max}}}$: látszólagos súrlódási félkúpszög \siunit{}{\degree} \\
		${M_\text{anya}}_{\substack{\text{min}\\\text{max}}}$: csavaranya felülete alatti súrlódás \siunit{}{\newton\mm} \\
		${M_\text{meghúzási}}_{\substack{\text{min}\\\text{max}}}$: meghúzási nyomaték \siunit{}{\newton\mm} \\
	\end{tabular}
\end{center}
